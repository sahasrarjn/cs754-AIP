\documentclass{article}



\usepackage{arxiv}

\usepackage[utf8]{inputenc} % allow utf-8 input
\usepackage[T1]{fontenc}    % use 8-bit T1 fonts
\usepackage{hyperref}       % hyperlinks
\usepackage{url}            % simple URL typesetting
\usepackage{booktabs}       % professional-quality tables
\usepackage{amsfonts}       % blackboard math symbols
\usepackage{nicefrac}       % compact symbols for 1/2, etc.
\usepackage{microtype}      % microtypography
\usepackage{lipsum}		% Can be removed after putting your text content
\usepackage{graphicx}
\usepackage{natbib}
\usepackage{doi}



\title{CS754 - Robust Video Denoising}

\date{May 11, 2021}	% Here you can change the date presented in the paper title

\author{ \href{https://orcid.org/0000-0000-0000-0000}{\includegraphics[scale=0.06]{orcid.pdf}
    \hspace{1mm}Rahul Prajapat}\\
	Department of Computer Science and Engineering\\
	Indian Institute of Technology, Bombay\\
	\texttt{190050095@iitb.ac.in} \\
	%% examples of more authors
	\And
	\href{https://orcid.org/0000-0000-0000-0000}{\includegraphics[scale=0.06]{orcid.pdf}
	\hspace{1mm}Sahasra Ranjan} \\
	Department of Computer Science and Engineering\\
	Indian Institute of Technology, Bombay\\
	\texttt{190050102@iitb.ac.in} \\

}

% Uncomment to remove the date
%\date{}

% Uncomment to override  the `A preprint' in the header
%\renewcommand{\headeright}{Technical Report}
%\renewcommand{\undertitle}{Technical Report}
\renewcommand{\shorttitle}{\textit{arXiv} Template}

%%% Add PDF metadata to help others organize their library
%%% Once the PDF is generated, you can check the metadata with
%%% $ pdfinfo template.pdf
\hypersetup{
pdftitle={Robust Video Denoising},
pdfsubject={q-bio.NC, q-bio.QM},
pdfauthor={Rahul Prajapat, Sahasra Ranjna},
pdfkeywords={Video Denoising, Adaptive Median Filter, TSCS, TSS, Block Matching},
}

\begin{document}
\maketitle

\begin{abstract}
	Most existing video denoising algorithms assumes a single noise model but which is often violated in practice. We are presenting a patch based video denoising algorithm based on grouping similar patches in spatial and temporal domain. We will formulate this as low-rank matrix completion matrix. Our implementation based on the paper [1] will be effective for removing mixed noise in video sequences.
\end{abstract}


% keywords can be removed
\keywords{Video Denoising \and Adaptive Median Filter \and TSCS \and TSS \and Block Matching \and Block Motion Vectors}


\section{Introduction}
In general, video data tend to be more noisy than single image due to high speed capturing rate of video camera. Video denoising aims at efficiently removing noise from all frames of a video by utilizing information in both spatial and temporal domains. Standing out from most of the denoising techniques, our implementation will be able to perform well for video frames with mixed noise (viz. Gaussian Noise, Quantization Noise, Photon Shot noise, Inpulsive noise, \dots).


\section{Implementation}





\bibliographystyle{unsrtnat}
\bibliography{references}  %%% Uncomment this line and comment out the ``thebibliography'' section below to use the external .bib file (using bibtex) .


\begin{thebibliography}{9}
\bibitem{robustdenoise} 
Hui Ji, Chaoqiang Liu, Zuowei Shen and Yuhong Xu.
\textit{Robust video denoising using Low rank matrix completion.}

\bibitem{medianfilter}
H. Wang, R. A. Haddad
\textit{Adaptive median filters: New algorithms and research}

\bibitem{patchmatching}
Bede Liu, Andr\'e Zaccarin
\textit{New fast algorithms for the Estimation of Block motion vectors}

% add here @rahul

\end{thebibliography}



\end{document}
